\chapter{Use cases}

For illustrating the functionality of the application, we propose the following three use cases:

\section{First use case}

\begin{center}
	\centering
	\includegraphics[width = 5.5in]{images/usecase1.png}
	\centerline{\captionof{Figure 4.1: Illustration of the behaviour when the user inputs a correct link.}}
	\label{uc1}
	\end{center}
After inserting a valid link, it is inserted into the main application pipeline and the
behaviour is the expected one.
\section{Second use case}
\begin{center}
	\centering
	\includegraphics[width = 5.5in]{images/usecase3.png}
	\centerline{\captionof{Figure 4.2: Illustration of the behaviour when the user inputs an invalid link. }}
	\label{uc3}
	\end{center}
The link introduced by the user is invalid. The youtube-dl library will throw an exception and the execution will ceases, given that the input cannot be fetched.

\section{Third use case}

\begin{center}
	\centering
	\includegraphics[width = 5.5in]{images/usecase2.png}
\centerline{\captionof{Figure 4.3: Illustration of the behaviour when the user inputs a confusing song to the application.}}
	\label{uc2}
	\end{center}
 The input is confusing (e.g. a symphonic concert containing multiple other instruments beside a piano). Given that the model has been trained on piano-only samples, the prediction for such an input will be 'Other'.
