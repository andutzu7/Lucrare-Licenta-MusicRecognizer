\chapter*{Final conclusions and possible improvements}
\addcontentsline{toc}{chapter}{Conclusions and possible improvements}
\section*{Possible Improvements}
Given that the problem the application solves is a binary one, a possible improvement is to increase
the number of classes in order to represent multiple instruments.
One possible scenario would be distinguishing between the instruments of an orchestra.
Obviously that implies creating the corresponding 3D animations.\\
On the same note another improvement would refer to implementing GPU numeric operations support in
order to improve the speed of the training process which is known to be lengthy in
convolutional learning.
\section*{Final Conclusions}
The three constituent parts of the application (as presented in Chapter 4 of the current paper) form a project that serves its purpose: to both create a middle way between arts and computer science, making the applications of machine learning interesting, yet engaging and to get  a deeper understanding, by implementing the neuronal network framework from scratch, of the mathematical and theoretical subtleties of artificial intelligence.

		The application succeeds in creating the liaison between an intuitive and friendly user experience and the complex background of the custom machine learning framework. Therefore, we believe that the current thesis corresponds with the initial ambition of the project.
