\chapter*{Introduction}
\addcontentsline{toc}{chapter}{Introduction}

The current thesis is an attempt to materialize the intersection of artificial intelligence and arts, especially, music. \\

The application consists of two major parts:\\
	\begin{itemize}
	\addtolength{\itemindent}{1cm}
		\item  First, the machine learning component, represented by a neuronal network framework, implemented from scratch (i.e. without the explicit support of existent frameworks or libraries). The module presents the necessary functionalities for constructing a neuronal network: layers, activation functions, metrics, optimizers, models and support for data generation and preprocessing.\\

		\item Second, the visual support, compound of two illustrative animations, created in Autodesk Maya 2019, and a Graphical User Interface, in which a hypothetical user can interact with the application by feeding it a YouTube link of a song. As a result, the application will decide whether the input song is played on a piano or not (prediction Piano/Other). Depending on the decision of the neuronal network, the corresponding 3D animation will be played.\\
	\end{itemize}

	\section{Context}

	The first analogy between the way computers can process information and the way the human brain works (as we know, at least, to this day), has been made by Warren McCullough and Walter Pitts in 1944, who later became the founding members of what is sometimes called the first cognitive science department \cite{mitnn}. The primary idea is elegant in its simplicity: the neuron, the basis of the human cognitive apparatus, can be modelled in machine learning as an unit in a network.

	Since then, numerous studies and breakthroughs have been made, as well as various frameworks and tools which make implementing a neuronal network accessible without possessing the full mathematical background needed prior.

	\section{Purpose and modus operandi}

	One of the purposes of this thesis is to implement the functionalities and the logic behind a neuronal network, as well as creating and fine-tuning a model. In order to complete this task, we consulted multiple sources( e.g. \cite{nnfs},\cite{DLFMIR},\cite{BACDNN} ) which presented the theoretical and mathematical aspects of constructing the aforementioned classifier. The implementation was constructed with the support of various tools from the Python programming language (e.g. Numpy, Librosa, Tenserflow Keras, Kapre), which will be extensively discussed in the following chapters.

	As discussed before, the second component of the thesis regards the visual part of the application. For obtaining an interactive use of the project, we used Autodesk Maya 2019 (for creating the animations) and Python tkinkter for creating a simplistic GUI.

	The logical flow of the pipeline is as follows: the user feeds a YouTube link of a song to the GUI. The model (previously trained and saved) computes a prediction regarding the category under which the input falls (Piano/Other). Given this result, the corresponding animation is played.

	\section{Personal contributions}


	Numerous attempts of creating a medium between artificial intelligence and other disciplines have been made since the rise of this field. Arts, especially music, is no exception.

	The particularity of the current thesis is the approach we had in completing the task: implementing from scratch the neural network framework, and, implicitly, understanding the mathematical and theoretical subtleties of it, as well as creating the visual aid which aims to touch on (although briefly) 3D animations.

	\section{Structure}
	The structure of the present thesis follows the major constituent parts described before and is as follows:
	\begin{enumerate}
	\item Chapter One
		\begin{itemize}
			\item Similar applications :  in which other akin projects are mentioned;
		\end{itemize}

	\item Chapter Two
		\begin{itemize}
			\item Technologies used : in which technologies needed for creating the application and their purpose and functionalities are discussed;

		\end{itemize}
	\item Chapter Three
		\begin{itemize}
		\item Architecture and implementation : in which the actual implementation is discussed explicitly. This section contains technical and theoretical aspects of the thesis.

		\end{itemize}
	\item Chapter Four
		\begin{itemize}
			\item Use cases: in which a part of the functionalities of the project are presented;

		\end{itemize}
	\end{enumerate}

 	\section{Acknowledgements}


	I would like to express my special thanks of gratitude to the academical staff of the Faculty of Computer Science Iași for the opportunity to do this project.\\

	Obviously, the present thesis would not be possible without the help and guidance of the professor that directed it, PhD. Anca Vitcu.
