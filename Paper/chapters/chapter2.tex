\chapter{Technologies used}
Python provides many modules for numeric calculus, data processing and manipulation as  well as tools for user interaction.
In the following chapter we will go through the modules and libraries used for achieving the goal
of the application, instrument classification.

\section{Scipy}
SciPy is a Python-based ecosystem of open-source software for mathematics, science, and engineering. The SciPy library contains a collection of numerical algorithms and domain-specific toolboxes, including signal processing, optimization, statistics, and much more.\cite{scpd}
The aforemetioned library's submodule \textit{Input and output} provides a variety of ways of reading and writing data in numerous file formats. For the data processing part of the application, we had choosen the \textit{scipy.io.wavfile.read} function because it facilitates the data manipulation process by being compatible with the numpy module, mentioned below (the output of the \textit{scipy.io.wavfile.read} function is a tuple consisting of the sample rate(int) and the audio data(as a \textit{ndarray}).
\section{NumPy}
Numpy is a Python library (part of the SciPy ecosystem) that provides the needed tools in order to perform the mathematical operations behind
the Neural Network Framework. The core of the module is the \textit{ndarray} object, which encapsulates an optimized n-dimensional array. Thus, we use \textit{ndarray} objects for computing and storing the various operations performed throuought the application (e.g. the dot product, the data reshaping).
Some of the advantages brought by the numpy module are:
\begin{itemize}
	\item Forward and backward operation compatibility for the various objects contained withing the application
	\item Fast execution time caused by the optimization of the library \cite{npd} (by using pre-compiled C code)
	\item Support for large number/big data processing.
\end{itemize}

\section{TensorFlow}

\section{Kapre}
composed for the input layer

\section{Librosa}

\section{Youtube_dl}

\section{Tkinkter}
