\chapter{Technologies used}
Python provides many modules for numeric calculus, data processing and manipulation as  well as tools for user interaction.
In the following chapter we will breafly touch on the modules and libraries used for achieving the goal
of the application, instrument classification.

\section{Scipy}
SciPy is a Python-based ecosystem of open-source software for mathematics, science, and engineering. The SciPy library contains a collection of numerical algorithms and domain-specific toolboxes, including signal processing, optimization, statistics, and much more.\cite{scpd}
The aforemetioned library's submodule \textit{Input and output} provides a variety of ways of reading and writing data in numerous file formats. For the data processing part of the application, we had choosen the \textit{scipy.io.wavfile.read} function because it facilitates the data manipulation process by being compatible with the numpy module, mentioned below (the output of the \textit{scipy.io.wavfile.read} function is a tuple consisting of the sample rate(int) and the audio data(as a \textit{ndarray}).
\section{NumPy}
Numpy is a Python library (part of the SciPy ecosystem) that provides the needed tools in order to perform the mathematical operations behind
the Neural Network Framework. The core of the module is the \textit{ndarray} object, which encapsulates an
optimized n-dimensional array. Thus, we use \textit{ndarray} objects for computing and storing the various
operations performed throughout the application (e.g. the dot product, the data reshaping). The \textit{ndarray}
array is represented as a matrix, whose dimensions are referred as \textit{shape}.
Some of the advantages brought by the numpy module are:
\begin{itemize}
	\item Forward and backward operation compatibility for the various objects contained withing the application
	\item Fast execution time caused by the optimization of the library \cite{npd} (by using pre-compiled C code)
	\item Support for large number/big data processing.
\end{itemize}

\section{TensorFlow}
TensorFlow is an end-to-end open source platform for machine learning. It has a comprehensive, flexible ecosystem of tools, libraries and community resources that lets researchers push the state-of-the-art in ML and developers easily build and deploy ML powered applications.\cite{tf}
From the vast amount of methods and submodules contained within TensorFlow, we used Keras, an utilitary library
built on top of the TensorFlow API. In order for the Neural Network Framework to be able to handle large amounts
of data, we inherited in our custom \textit{DataGenerator class} the \textit{tensorflow.keras.utils.Sequence}
class. The \textit{Keras DataGenerator class} along with the custom Neural Network Framework \textit{DataGenerator}
provide among other functionalities the \textit{ getitem} method, which yields a batch of data at a given index, thus overcoming the big data handling problem.

\section{Kapre}
Kapre (Keras Audio Preprocessors) is an audio data preprocessing library built on top of \textit{Keras},
specialized on audio signal handling.\cite{choi2017kapre}
Kapre facilitates sound transformation by efficiently performing various operations (e.g. \textit{Short
Time Fourier Transformation}, \textit{Magnitude} ) in a GPU optimized consistent manner . For the input data modeling we had chosen the
\textit{get mel spectogram layer} function from the module \textit{kapre.composed}. It applies sequentially
\textit{STFT, Magnitude and mel filterbank} transformations over the given input data, returning a mel spectogram
with an user specified sample rate and output shape.
\section{Librosa}
Librosa is a python package for music and audio analysis. It provides the building blocks necessary to create music information retrieval systems. \cite{mcfee2015librosa}
Out of the Librosa API we have choosen the following tools :
\begin{itemize}
	\item \textit{resample}: A function that resamples an audio file from its original rate to an user
		given sample rate.

	\item \textit{to mono}: A function which converts a multi channel signal to mono signaling by averaging
		the sample values.
\end{itemize}
\section{youtube-dl}
Youtube-dl is an open source tool for downloading videos and songs from the YouTube platform. In the
context of the final application, the youtube-dl module provides the end user with the possibilty of classifying
a song just by inserting a link to a YouTube video.

\section{Tkinkter}
Tkinkter is the standard Python GUI Toolkit, a simple yet versatile module was used for the minimalistic Graphic
User Interface of the final application.
